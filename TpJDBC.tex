\documentclass[a4paper,11pt]{article}
\usepackage[utf8]{inputenc}
\usepackage[T1]{fontenc}
\usepackage[x11names,table]{xcolor}
\usepackage[french]{babel}
\usepackage{wasysym}
\usepackage{natbib}
% Si l'on veut produire une version PDF avec distiller ou pdflatex:
\usepackage[pageanchor=false,colorlinks,plainpages=false]{hyperref}
\usepackage{url}

\ifx\pdftexversion\undefined
\usepackage[dvips]{graphicx}
\else
\usepackage[pdftex]{graphicx}
\DeclareGraphicsRule{*}{mps}{*}{}
\fi

\graphicspath{{Images/}}

\usepackage{time}
%\usepackage[scaled]{helvet}
%\renewcommand*\familydefault{\sfdefault} %% Only if the base font of the document is to be sans serif

\usepackage{listings}
\usepackage{todonotes}
\usepackage{tikz-er2}
\usepackage{pgf-umlcd}
\usetikzlibrary{decorations.text}
\usetikzlibrary{decorations.markings}
\usetikzlibrary{positioning}
\usetikzlibrary{shadows}
\usetikzlibrary{backgrounds}
\tikzstyle{every entity} = [top color=white, 
                            bottom color=blue!30, 
                            draw=blue!50!black!100, 
                            drop shadow]
\tikzstyle{every weak entity} = [drop shadow={shadow xshift=.7ex, shadow yshift=-.7ex}]
\tikzstyle{every attribute} = [top color=white, 
                               bottom color=yellow!20, 
                               draw=yellow, 
                               node distance=1cm, 
                               drop shadow]
\tikzstyle{every relationship} = [top color=white, 
                                  bottom color=yellow!20,  
                                  draw=yellow, 
                                  drop shadow]
\tikzstyle{every isa} = [draw=blue!50!black!100]

\usepackage[]{subfig}
%\renewcommand{\thesubfigure}{Figure~\thefigure.\arabic{subfigure}}
%\captionsetup[subfigure]{labelformat=simple,labelsep=colon, listofformat=subsimple}
%\captionsetup{lofdepth=2}
%\makeatletter
%\renewcommand{\p@subfigure}{}
%\makeatother

\usepackage{geometry}
\geometry{%
a4paper,
body={160mm,250mm},
left=25mm,top=20mm,
headheight=7mm,headsep=4mm,
marginparsep=3mm,
marginparwidth=27mm}

\usepackage{changepage}
\usepackage{placeins}

\usepackage{rotating}

\newenvironment{agrandirmarges}[2]{%
\begin{list}{}{%
\setlength{\topsep}{0pt}%
\setlength{\listparindent}{\parindent}%
\setlength{\itemindent}{\parindent}%
\setlength{\parsep}{0pt plus 1pt}%
\checkoddpage%
\ifoddpage
\setlength{\leftmargin}{-#1}\setlength{\rightmargin}{-#2}
\else
\setlength{\leftmargin}{-#2}\setlength{\rightmargin}{-#1}
\fi}\item }%
{\end{list}}

\newcounter{compteurQuestion}
\setcounter{compteurQuestion}{0}
\newcommand{\Question}{\paragraph*{Question~\thecompteurQuestion~:}\addtocounter{compteurQuestion}{1}}

\date{}
\begin{document}
\lstset{
         breaklines=true,                                     % line wrapping on
         %frame=ltrb,
         framesep=5pt,
         %samepage=true,
         tabsize=4,
         basicstyle=\normalsize,
         frameround=ftft,
         keywordstyle=\ttfamily\color{SeaGreen4},
         identifierstyle=\ttfamily\bfseries\color{RoyalBlue4},
         commentstyle=\color{RoyalBlue3},
         stringstyle=\ttfamily,
         showstringspaces=false
}
\pgfdeclarelayer{background}
\pgfdeclarelayer{foreground}
\pgfsetlayers{background,main,foreground}

\newlength{\niveauZero}
\newlength{\niveauUn}
\newlength{\niveauDeux}
\newlength{\niveauTrois}
\newlength{\niveauQuatre}
\newlength{\niveauCinq}

\newlength{\colonneZero}
\newlength{\colonneUn}
\newlength{\colonneDeux}
\newlength{\colonneTrois}
\newlength{\colonneQuatre}
\newlength{\colonneCinq}


{\centering
    \mbox{
      \makebox[15cm][l]{
      \begin{minipage}{15cm}
        \begin{center}
          {\Huge Construction d'une Couche de Persistance}\\[1cm]
          {\Large Sébastien \textsc{Nedjar} et Fabien \textsc{Pesci}}
        \end{center}
      \end{minipage}
      }
    }
}\\[0.4cm]
\section{Introduction}

Le principal défaut que l'on peut reprocher à JDBC est d'être une API de bas niveau qui conduit à une trop forte 
imbrication entre le code métier et la base de données. Le code produit est donc trop peu modulaire et 
trop dépendant du \textsc{Sgbd} choisit. Cela implique une moins grande maintenabilité et une plus grande 
dépendance face à une technologie de manipulation des données. Pour contourner cette difficulté, nous allons construire 
une couche dédiée à l'accès aux données  (souvent appelée couche DAO). La construction d'une telle couche a pour objectif de séparer totalement 
les accès aux données du code de notre application. Les techniques présentées constituent une première introduction par 
la pratique aux solutions de persistance Objets/Relationnelle comme Hibernate\footnote{\url{http://www.hibernate.org/}} 
ou EclipseLink\footnote{\url{http://www.eclipse.org/eclipselink/}}.

Pour illustrer ce propos, nous utiliserons la base de données «~Gestion Pédagogique\footnote{Script de régénération disponible 
à l'adresse suivante : \url{http://allegro.iut.univ-aix.fr/~nedjar/gestion_peda_oracle.sql} ou 
\url{http://allegro.iut.univ-aix.fr/~nedjar/gestion_peda_mysql.sql}}~» que vous avez utilisée lors de vos TP de \textsc{Pl/Sql} 
en début d'année. Le modèle conceptuel des données est rappelé par la figure~\ref{mcd_gestion_peda}.

\begin{figure}[bp]\centering
\begin{tikzpicture}[node distance=1.96cm, every edge/.style={link}]

  \node[entity] (mat) {
    \textbf{Module}
    \nodepart{second}
    \key{Code}\\
    Libellé\\
    H\_Cours\_Prev\\
    H\_Cours\_Rea\\
    H\_TP\_Prev\\
    H\_TP\_Rea\\
    Discipline\\
    Coef\_Test\\
    Coef\_CC\\
  };

  \node[entity] (etud) [below right =of mat ] {
    \textbf{Etudiant}
    \nodepart{second}
    \key{Num\_Et}\\
	  Nom\_Et\\
	  Prénom\_Et\\
	  CP\_Et\\
	  Ville\_Et\\
	  Année\\
	  Groupe\\
  };

  \node[entity] (ens) [above right=of etud ] {
    \textbf{Prof}
    \nodepart{second}
    \key{Num\_Prof}\\
	  Nom\_Prof\\
	  Prénom\_Prof\\
	  Adr\_Prof\\
	  CP\_Prof\\
	  Ville\_Prof\\
  };

  \node[relationship] (notation) [below =of mat] {Notation}
     child {node[attributes] {Moy\_CC\\Moy\_Test}};
  \draw[link] (mat) -- node [pos=0.15, auto] {(0,n)} (notation);
  \draw[link] (etud.145) -- node [pos=0.35, auto, swap] {(0,n)} (notation);

  \node[relationship] (enseigne) [above=of etud] {Enseignement};  
  \draw[link] (mat.315) -- node [pos=0.40, auto] {(0,n)} (enseigne.west);
  \draw[link] (ens.208) -- node [pos=0.40, auto, swap] {(0,n)} (enseigne.east);
  \draw[link] (etud) -- node [pos=0.15, auto, swap] {(0,n)} (enseigne);

  \node[relationship] (mat_spec) [above=1cm of enseigne] {Spécialiste};
  \draw[link] (mat.12) -- node [pos=0.35, auto] {(0,n)} (mat_spec);
  \draw[link] (ens.145) -- node [pos=0.35, auto, swap] {(1,1)} (mat_spec);
  
  \node[relationship] (resp) [above=1cm of mat_spec] {Responsable};
  \draw[link] (mat.56) -- node [pos=0.3, auto] {(1,1)} (resp);
  \draw[link] (ens.north) |- node [pos=0.15, auto, swap] {(0,n)} (resp.east);

  \node[relationship] (mat_pere) [left=1cm of mat] {A pour père};
  \draw[link] (mat.150) -| node [pos=0.1, auto, swap] {(1,1)} (mat_pere.north);
  \draw[link] (mat.210) -| node [pos=0.1, auto] {(0,n)} (mat_pere.south);
  
\end{tikzpicture}
\caption{Modèle conceptuel des données de la base «~Gestion Pédagogique~»\label{mcd_gestion_peda}}
\end{figure}

\section{Couche de persistance}
L'API JDBC permet de facilement récupérer et manipuler un ensemble de tuples récupéré à partir d'une base de données. 
Chaque tuple n'est pas simplement  une concaténation de valeurs sans rapport les unes avec les autres mais un ensemble 
de valeurs structuré permettant de modéliser une «~entité~» de l'univers réel. Lorsqu'un tuple est récupéré à partir de 
JDBC, il faut donc impérativement conserver ce  lien sémantique existant entre les attributs. C'est pour cela que chaque 
tuple de la base  de données devra être associé (mappé) à un objet du langage de programmation.  Une fois le mapping 
établi, l'objet commence son existence autonome comme n'importe quel  autre objet de l'application.  Son état (ensemble 
des valeurs de ses propriétés) sera très probablement mis à jour.  Afin que ces changements soient visibles pour les 
autres utilisateurs, il faudra  périodiquement synchroniser l'état de l'objet et de la base de données. Une fois les  
modifications sauvegardées, l'objet pourra être détruit car l'utilisateur peut à tout moment reconstruire un objet 
semblable à partir de la base de données. Le mapping objet/relationnel  permet ainsi de rendre les objets de 
l'application persistants.

Dans la suite de cette section nous allons montrer une méthode pour créer «~proprement~»  un tel mapping. La solution 
présentée est principalement pédagogique : elle ne sera en conséquence pas satisfaisante pour une solution déployée 
à grande échelle, mais sera amplement suffisante  pour le développement d'une application mono-utilisateur (comme celle 
que vous devez réaliser en \emph{Cas 2}).

\subsection{Connexion à la base de données}
Recréer une connexion pour chaque requête est inutile et coûteux. Pour éviter cela, il faut  partager la connexion entre
plusieurs traitements. La problématique est de savoir quelles sont les  requêtes à exécuter ensemble. La solution 
classique est d'exécuter au sein d'une même connexion  tous les traitements concourant à la réalisation d'un même 
objectif. Chacun de ces ensembles  de traitements constitue une session. Une même session peut contenir plusieurs 
transactions  (unités de traitement indivisibles)\footnote{\url{http://fr.wikipedia.org/wiki/Transaction_informatique}}.
Tout ceci permet de gérer efficacement et intelligemment  les problèmes de concurrence et de reprise après erreur. 
Réaliser une gestion réaliste des  connexions, sessions et transactions demande un travail important que nous n'avons
pas les  moyens fournir. Notre application étant simple et mono-utilisateur nous utiliserons la méthode  dite 
\emph{«~session-per-application~»}, c'est à dire qu'il n'y aura qu'une seule connexion active  à la fois et tous les 
objets devront se la partager. Le pattern  singleton\footnote{\url{http://fr.wikipedia.org/wiki/Singleton_\%28patron_de_conception\%29}} 
est mis en œuvre pour  que tous les objets de notre application puissent récupérer l'unique instance de la classe 
\texttt{Connection}.

\begin{figure}\centering
\begin{tikzpicture}
\begin{singleton}[text width=6cm]{ConnexionUnique}{0,0}

\attribute{-connection : Connection}
\attribute{-\underline{instance} : ConnexionUnique }
\attribute{-\underline{CONNECT\_URL} : String }
\operation{-ConnexionUnique()}
\operation{+getConnection() : Connection }
\operation{+\underline{getInstance()} : ConnexionUnique }

\end{singleton}
\end{tikzpicture}
\caption{Classe \texttt{ConnexionUnique}\label{connexion}}
\end{figure}

\Question
Écrire la classe \texttt{ConnexionUnique} dont le diagramme UML vous est donné dans la figure~\ref{connexion}.
Copier la classe \texttt{testJDBC} dans la nouvelle classe \texttt{testConnexion}.
Modifier le code de cette nouvelle classe pour qu'elle utilise notre objet \texttt{ConnexionUnique}.

\subsection{Création des classes d'objets métiers}
La création d'un mapping entre le «~monde objet~» et le «~monde relationnel~» nécessite au préalable la création de 
modèles de données semblables mais adaptés aux spécificités de chacun de ses mondes.  L'objectif est donc de transformer 
le modèle relationnel de la base «~Gestion Pédagogique~» (schéma Entité/Association)  en un modèle objet satisfaisant 
(diagramme de classes UML\footnote{\url{http://uml.free.fr/}}).

La figure~\ref{uml_gestion_peda} est une traduction directe du schéma Entité/Association en un  diagramme de classe UML. 
Chacun des concepts du schéma E/A a été transformé en son équivalent UML \footnote{Le losange est le symbole 
matérialisant les associations n-aires ($n > 2$). \texttt{Notation} est une classe dite d'association. Elle permet de 
modéliser les attributs portés par une association.}. À partir de cette traduction, nous allons montrer les 
modifications à apporter à ce modèle pour le rendre implémentable.

\begin{figure}\centering

\begin{tikzpicture}[]

\begin{class}[text width=2cm]{Module}{0,0}
\end{class}

\begin{class}[text width=2cm]{Prof}{10,0}
\end{class}

\begin{class}[text width=2cm]{Etudiant}{5,-4.5}
\end{class}

\begin{class}[text width=2cm]{Notation}{0,-6}
\end{class}

\draw[rounded corners=1em,color=\umldrawcolor,text=\umltextcolor,decoration={text along path, text={Est pere de}, text align=center}, decorate]
(Module.west) -| +(-1.1,1.3) -| (Module.north);

\draw [rounded corners=1em,color=\umldrawcolor,text=\umltextcolor] 
(Module.west) -| +(-1,1.2) node[near start, above]{1} -| (Module.north) node[near end, right]{*};

\draw [color=\umldrawcolor,text=\umltextcolor] (Prof.160) to 
node[very near start, above]{1}
node [above]{A pour responsable $\RHD$}
node[very near end, above]{*}  
(Module.20);

\draw [color=\umldrawcolor,text=\umltextcolor] (Module.340) to
node[very near start, above]{1}
node [below]{$\LHD$ Est spécialiste}
node[very near end, above]{*}
(Prof.200);

\draw [color=\umldrawcolor,text=\umltextcolor] (Module.south) |- node[](a){} (Etudiant.west)
node[very near start, left]{*}
node[very near end, above]{*};

\draw [fill,color=\umldrawcolor,dashed] (a) circle (2pt)--(Notation);

\ternaryassociation%
{Module.south}%
{Prof.south}%
{Etudiant}%
{Enseignement}
\end{tikzpicture}
\caption{Traduction UML du modèle conceptuel des données de la base «~Gestion Pédagogique~»\label{uml_gestion_peda}}
\end{figure}

\subsubsection{Modélisation des types d'entités}
Pour établir une correspondance entre une entité de notre BD et un objet de notre application, il faut commencer par 
écrire les classes associées à chacun des types d'entités. Pour des raisons qui apparaitront plus tard, chaque classe 
métier devra suivre les conventions suivantes :
\begin{itemize}
	\item  La classe doit être «~sérialisable~»\footnote{\url{http://fr.wikipedia.org/wiki/S\%C3\%A9rialisation}} 
	(\emph{i.e.} implémenter l'interface \texttt{Serializable}) pour pouvoir sauvegarder et restaurer l'état des instances 
	de cette classe ;
  \item  La classe doit posséder un constructeur sans argument (constructeur par défaut);
  \item  Les propriétés privées de la classe (variables d'instances) doivent être accessibles publiquement via des 
  méthodes accesseurs construites avec \texttt{get} ou \texttt{set} suivi du nom de la propriété avec la première lettre 
  transformée en majuscule (voir le menu \texttt{Source $\rightarrow$ Generate Getters and Setters...} dans Eclipse).
  \item  La classe doit surcharger la méthode \texttt{toString()} pour pouvoir afficher l'état des instances de cette 
  classe (voir le menu \texttt{Source $\rightarrow$ Generate toString()} dans Eclipse).
  \item  La classe doit aussi surcharger les méthodes \texttt{equals()} et \texttt{hashCode()} héritées de 
  \texttt{Object}\footnote{\url{http://download.oracle.com/javase/6/docs/api/java/lang/Object.html}} 
  (voir le menu \texttt{Source $\rightarrow$ Generate hashCode() and equals()...} dans Eclipse). Deux objets métiers
  sont égaux si et seulement si les tuples associés sont égaux dans la base de données (\emph{i.e.} même valeur de clef). 
\end{itemize}

\Question
Implémenter (en respectant les conventions ci-dessus) les classes \texttt{Etudiant}, \texttt{Module} et \texttt{Prof} dont le diagramme 
UML incomplet vous est donné dans la figure~\ref{classe_metier}. Copier la classe \texttt{testJDBC} dans la nouvelle 
classe \texttt{testEntite}. Modifier le code de cette classe pour que sa boucle principale remplisse un 
\texttt{ArrayList}\footnote{\url{http://docs.oracle.com/javase/6/docs/api/java/util/ArrayList.html}} d'objets 
\texttt{Etudiant} et qu'elle affiche le contenu de cette liste en utilisant la méthode \texttt{toString()}.

\begin{figure}\centering
\begin{tikzpicture}[]

\begin{class}[text width=3.5cm]{Module}{0,0}
\attribute{-code : String }
\attribute{-libelle : String}
\attribute{-hCoursPrev : int}
\attribute{-hCoursRea : int}
\attribute{-hTpPrev : int}
\attribute{-hTpRea : int}
\attribute{-discipline : String}
\attribute{-coefTest : int}
\attribute{-coefCc : int}
\end{class}

\begin{class}[text width=4cm]{Prof}{5,0}
\attribute{-numProf : int}
\attribute{-nomProf : String}
\attribute{-prenomProf : String}
\attribute{-adrProf : String}
\attribute{-cpProf : String}
\attribute{-villeProf : String}
\end{class}

\begin{class}[text width=3.5cm]{Etudiant}{10,0}
\attribute{-numEt : int}
\attribute{-nomEt : String}
\attribute{-prenomEt : String}
\attribute{-cpEt : String}
\attribute{-villeEt : String}
\attribute{-annee : int}
\attribute{-groupe : int}
\end{class}
\end{tikzpicture}
\caption{Classe \texttt{Etudiant}, \texttt{Module} et \texttt{Prof}\label{classe_metier}}
\end{figure}

\subsubsection{Modélisation des types d'association hiérarchiques}
Dans la figure~\ref{uml_gestion_peda} tous les types d'association binaires du MCD ont été représentés 
par leurs équivalents en UML~\citep{book/Booch1999,book/Rumbaugh1998}. Ces associations UML sont symbolisées par un trait liant deux classes. 
Les multiplicités (les nombres situés aux extrémités de l'association) correspondent aux cardinalités 
du MCD mis à part qu'elles sont placées à l'inverse. Par exemple, pour indiquer qu'une classe \texttt{A}
peut participer 0 ou 1 fois à une association avec la classe \texttt{B}, on placerait la 
multiplicité \texttt{0..1} du coté de \texttt{B}. UML permet d'écrire certaines multiplicités de 
manière simplifiée : \texttt{0..*} devient \texttt{*} et \texttt{1..1} devient \texttt{1}.

\begin{figure}\centering

\begin{tikzpicture}[]

\begin{class}[text width=2cm]{Module}{0,0}
\end{class}

\begin{class}[text width=2cm]{Prof}{9,0}
\end{class}

\draw [color=\umldrawcolor,text=\umltextcolor, rounded corners=2em, 
       decoration={markings,mark=at position 1 with {\arrow[scale=2]{angle 60}}},
       postaction={decorate}, shorten >=0.4pt] 
       (Module.north)  to ++(0,1) to node[above] {Est père de $\RHD$} ++(-2.5,0) to ++(0,-1.5) to node[pos=0.7, above]{1} (Module.west)
       node[very near end, above right]{*};

\draw [color=\umldrawcolor,text=\umltextcolor, bend left, 
       decoration={markings,mark=at position 1 with {\arrow[scale=2]{angle 60}}},
       postaction={decorate}, shorten >=0.4pt] 
(Module.20) to 
node[very near end, below]{1}
node [above]{A pour responsable $\RHD$}
node[very near start, below]{*}  
(Prof.160);

\draw [color=\umldrawcolor,text=\umltextcolor, bend left,
       decoration={markings,mark=at position 1 with {\arrow[scale=2]{angle 60}}},
       postaction={decorate}, shorten >=0.4pt]  
(Prof.200) to
node[very near end, above]{1}
node [below]{$\LHD$ Est spécialiste}
node[very near start, above]{*}
(Module.340);
\end{tikzpicture}
\caption{Association à navigabilité restreinte\label{uml_navigabilite}}
\end{figure}

Par défaut les associations sont bidirectionnelles, cela signifie qu'une instance à l'une 
des extrémités peut savoir avec quelles autres instances elle est liée par 
cette association. Dans la pratique, ce double lien peut être coûteux à maintenir, c'est 
pourquoi UML permet de privilégier un seul sens en interdisant l'accès dans l'autre. 
C'est ce que l'on appelle la restriction de la navigabilité d'une association. 
Elle est symbolisée par une flèche indiquant le sens de navigation permis. 

La figure~\ref{uml_navigabilite} montre le sens de navigation des trois types d'association 
hiérarchiques de la base «~Gestion Pédagogique~». Les sens de navigation choisis imposent que :  
\begin{itemize}
	\item pour chaque instance de la classe \texttt{Prof} 
  on connaitra le \texttt{Module} pour lequel il est spécialiste mais pour un \texttt{Module} on ne 
  peut pas savoir quels sont les \texttt{Prof} spécialistes;
  \item pour chaque instance de la classe \texttt{Module} 
  on connaitra le \texttt{Prof} responsable mais pour un \texttt{Prof} on ne 
  peut pas savoir quels sont les \texttt{Modules} dont il est responsable;
  \item pour chaque instance de la classe \texttt{Module} 
  on connaitra son \texttt{Module} père mais pour un \texttt{Module} donné on ne peut pas 
  retrouver l'ensemble de ses fils.
\end{itemize}

\Question Implémenter en respectant le sens de navigation imposé l'association 
\textit{«~est spécialiste~»} entre \texttt{Prof} et \texttt{Module}. 
Un objet \texttt{Prof} n'étant associé qu'à un seul \texttt{Module}, il suffit d'ajouter à 
la classe \texttt{Prof} un attribut \texttt{specialite} (sans oublier les accesseurs associés) qui est une référence vers un \texttt{Module}. 
Il permet de lier un objet \texttt{Prof} à sa spécialité.

Faire de même pour les deux autres TA hiérarchiques en respectant à chaque fois les sens de navigation de la
figure~\ref{uml_navigabilite}. Copier la classe \texttt{testJDBC} dans la nouvelle classe \texttt{testAsso1}. 
Modifier le code de cette classe pour remplir un \texttt{ArrayList} d'objets \texttt{Prof}. Pour chacun d'eux construire un objet
\texttt{Module} représentant sa spécialité et conserver une référence vers cet objet dans l'attribut
\texttt{specialite}. Afficher chacun des profs et le module dont il est spécialiste.


\subsubsection{Modélisation des types d'association non hiérarchiques}
Contrairement aux types d'association hiérarchiques qui peuvent être implémentés simplement par 
des références (pointeurs en $C++$), les types d'association non hiérarchiques nécessitent une 
structure supplémentaire~\citep{tse/Milicev2007,jot/Genova03,tse/Barbier2003}. Nous allons présenter trois manières d'implémenter ces 
types d'association : les collections de pointeurs de chaque coté de l'association, 
les objets d'association et la promotion d'une association en classe. Chacune de ces méthodes 
d'implémentation a des avantages et des inconvénients qu'il faudra prendre en compte avant 
de faire un choix.
\paragraph{Collections de pointeurs aux extrémités de l'association :}
La première méthode est en quelque sorte une extension de la technique 
d'implémentation du paragraphe précédent. Pour simplifier la présentation, cette approche est 
appliquée dans un premier temps sur l'association \texttt{Notation}, dans laquelle on ne considère pas les 
données portées par l'association (\emph{cf.} figure~\ref{uml_notation1}). L'implémentation complète (en 
rajoutant la classe d'association) de cette association sera faite dans un second temps.

\begin{figure}\centering
\begin{tikzpicture}
\begin{class}[text width=2cm]{Module}{0,0}
\end{class}
\begin{class}[text width=2cm]{Etudiant}{9,0}
\end{class}
\draw [color=\umldrawcolor,text=\umltextcolor] 
(Module.east) to 
node[very near end, below]{*}
node [above]{Notation}
node[very near start, below]{*}  
(Etudiant.west);
\end{tikzpicture}
\caption{Association \texttt{Notation} sans considérer la classe d'association\label{uml_notation1}}
\end{figure}

Dans le cas de l'association «~\texttt{est spécialiste}~» où un \texttt{Prof} n'était lié qu'à un seul
\texttt{Module}, il a suffi d'ajouter dans \texttt{Prof} une référence vers une instance de
\texttt{Module}. Ici, un \texttt{Etudiant} peut être lié à plusieurs \texttt{Module}. On ajoute donc non pas 
une seule référence, mais un ensemble (ou collection) de références, nommé \texttt{notations}, vers des objets \texttt{Module}. 
Cette collection doit être d'un type implémentant l'interface \texttt{Set}\footnote{Les spécificités de 
l'interface \texttt{Set} sont présentées sur la page suivante : 
\url{http://java.sun.com/developer/onlineTraining/collections/Collection.html\#SetInterface}} tel que
\texttt{HashSet} ou \texttt{TreeSet}. Cette contrainte garantit l'unicité des objets contenus dans la collection. 
Ainsi, un même \texttt{Etudiant} ne peut pas être lié plusieurs fois à un même \texttt{Module}, ce
qui indispensable pour modéliser correctement une association.

Aucun sens de navigation n'étant privilégié, il faut rajouter de manière symétrique une collection appelée 
\texttt{etudiants} dans \texttt{Module}. Cet ensemble de références vers des objets \texttt{Etudiant} rend 
possible la navigation dans le sens inverse. Le diagramme de la figure~\ref{classe_notation_set} décrit 
les changements apportés aux classes \texttt{Etudiant} et \texttt{Module} pour implémenter l'association.

\begin{figure}\centering\footnotesize
\begin{tikzpicture}[]
\begin{class}[text width=5.5cm]{Module}{0,0}
\attribute{-code : String }
\attribute{-libelle : String}
\attribute{-hCoursPrev : int}
\attribute{-hCoursRea : int}
\attribute{-hTpPrev : int}
\attribute{-hTpRea : int}
\attribute{-discipline : String}
\attribute{-coefTest : int}
\attribute{-coefCc : int}
\attribute{-responsable : Prof}
\attribute{-pere : Module}
\attribute{\textbf{-etudiants : Set<Etudiant>}}
\end{class}
%
\begin{class}[text width=5.5cm]{Etudiant}{10,0}
\attribute{-numEt : int}
\attribute{-nomEt : String}
\attribute{-prenomEt : String}
\attribute{-cpEt : String}
\attribute{-villeEt : String}
\attribute{-annee : int}
\attribute{-groupe : int}
\attribute{\textbf{-notations : Set<Module>}}
\end{class}
\end{tikzpicture}
\caption{Classes \texttt{Module} et \texttt{Etudiant} utilisant deux \texttt{Set}\label{classe_notation_set}}
\end{figure}

\begin{figure}\centering
\begin{tikzpicture}[]
\begin{class}[text width=2cm]{Module}{0,0}
\end{class}

\begin{class}[text width=2cm]{Etudiant}{8,0}
\end{class}

\begin{class}[text width=3cm]{Notation}{4,-2.5}
\attribute{-moyCC : float }
\attribute{-moyTest : float}
\end{class}

\draw [color=\umldrawcolor,text=\umltextcolor] (Module.east) -- node[](a){} (Etudiant.west)
node[very near start, above]{*}
node[very near end, above]{*};
\draw [fill,color=\umldrawcolor,dashed] (a) circle (2pt)--(Notation);
\end{tikzpicture}
\caption{Association \texttt{Notation} en considérerant les attributs portés\label{uml_notation2}}
\end{figure}

L'implémentation proposée permet de savoir à quel \texttt{Module} un \texttt{Etudiant} est lié (et inversement)
mais elle ne permet pas d'ajouter des informations supplémentaires aux liens. Pour implémenter l'association 
comme dans la figure~\ref{uml_notation2}, il faut prendre en compte la classe d'association \texttt{Notation}. 
Les ensembles de références sont remplacés par des dictionnaires (des conteneurs associatifs) pour atteindre cet objectif. 
Un dictionnaire peut être globalement perçu, d'un point de vue fonctionnel, comme une sorte de tableau 
indexable par n'importe quel type d'objet (plus seulement par des entiers). Malgré leur simplicité d'utilisation, ils ont 
un coût d'accès plus élevé qu'un tableau classique~\citep{book/Sedgewick1998}. En Java, les conteneurs associatifs 
sont des classes implémentant l'interface \texttt{Map}\footnote{Les spécificités de l'interface \texttt{Map} 
sont présentées à la page \url{http://java.sun.com/developer/onlineTraining/collections/Collection.html\#MapInterface}} 
(tel que \texttt{HashMap}). Ces classes permettent d'associer un objet clef (l'objet servant d'index) 
à un objet valeur (n'importe quel autre objet). D'après le diagramme de classe de la figure~\ref{uml_notation2}, cet 
objet valeur sera une référence vers un objet de la classe \texttt{Notation}. 
Les modifications à apporter aux classes \texttt{Module} et \texttt{Etudiant} pour prendre en compte ces changements sont décrites 
dans la figure~\ref{classe_notation_map}. Le lien \texttt{Notation} entre un \texttt{Etudiant} et un \texttt{Module} 
est ainsi représenté sous forme de collections (associatives) de pointeurs de part et d'autre de l'association.

\begin{figure}\centering\footnotesize
\begin{tikzpicture}[]
\begin{class}[text width=7cm]{Module}{0,0}
\attribute{-code : String }
\attribute{-libelle : String}
\attribute{-hCoursPrev : int}
\attribute{-hCoursRea : int}
\attribute{-hTpPrev : int}
\attribute{-hTpRea : int}
\attribute{-discipline : String}
\attribute{-coefTest : int}
\attribute{-coefCc : int}
\attribute{-responsable : Prof}
\attribute{-pere : Module}
\attribute{\textbf{-etudiants : Map<Etudiant, Notation>}}
\end{class}

\begin{class}[text width=7cm]{Etudiant}{8.5,0}
\attribute{-numEt : int}
\attribute{-nomEt : String}
\attribute{-prenomEt : String}
\attribute{-cpEt : String}
\attribute{-villeEt : String}
\attribute{-annee : int}
\attribute{-groupe : int}
\attribute{\textbf{-notations : Map<Module, Notation>}}
\end{class}

\end{tikzpicture}
\caption{Classe \texttt{Module}, \texttt{Etudiant} avec prise en compte des notations\label{classe_notation_map}}
\end{figure}

\paragraph{Objets d'association :}
L'approche précédente est relativement simple à mettre en œuvre du point de vue des 
modifications à apporter aux différentes classes. La principale difficulté provient de 
l'interdépendance entre objets qu'elle introduit. 
En effet, chacun des objets participant à une association a la responsabilité de 
construire et de maintenir à jour sa propre liste de liens. Si l'on souhaite supprimer un objet, 
il faut avant cela supprimer cet objet dans chacune des listes des objets avec lequel il 
est lié.  La responsabilité de la cohérence (réciprocité) d'un lien est partagée entre plusieurs 
objets de classes différentes, il y a donc éparpillement du code de gestion l'association ce qui implique 
un risque plus important d'erreur.

Dans notre cas (application avec objets persistants en BD) une telle approche n'est pas envisageable, 
car lorsque l'on doit rendre persistant un objet dans une base de données, cela implique de vérifier si les objets 
avec lesquels il est lié sont déjà stockés dans la base de données. Or, cette tache 
n'est pas du tout évidente d'un point de vue algorithmique et a un coût important s'il 
existe un grand nombre de liens.

La seconde solution~\citep{raool/Gessenharter2009} consiste à créer un unique objet qui aura la responsabilité de conserver 
et gérer tous les liens d'une association. Cet objet ayant une vision globale des liens existants, 
il peut facilement supprimer tous les liens entretenus par un seul et même objet. 
De plus, un objet n'a plus à connaitre tous les objets qui lui sont liés mais uniquement l'objet 
association qui pourra retrouver au besoin tous ces liens. En quelque sorte, cette approche 
est une solution globale qui décharge les différents objets de la responsabilité de gérer chacun 
des liens localement. Pour rendre persistante une association, il suffit de stocker tous les 
objets connus par l'association avant de stocker l'objet d'association lui même.

L'implémentation d'un objet d'association se fait en utilisant un  ensemble (\texttt{Set}) d'objet lien.
Chaque objet lien est un $n$-uplet de références vers les différentes classes participant à l'association.
La figure~\ref{classe_association_notation} donne le diagramme de classe de l'objet d'association \texttt{AssociationNotation}.
Un \texttt{Lien} est dans notre cas un triplet d'étudiant, module et notation. Pour gérer correctement la contrainte d'unicité 
de l'association, la classe \texttt{Lien} doit surcharger les méthodes \texttt{equals()} et \texttt{hashCode()} héritées de 
\texttt{Object}\footnote{\url{http://download.oracle.com/javase/6/docs/api/java/lang/Object.html}}. Deux liens sont 
considérés comme égaux s'ils référencent le même étudiant et le même module (peu importe la note).

\begin{figure}\centering\footnotesize
\begin{tikzpicture}
\begin{singleton}[text width=8cm]{AssociationNotation}{0,0}
\attribute{-liens : Set<Lien>}
\attribute{-\underline{instance} : AssociationNotation}

\operation{+creerLien(Module, Etudiant, Notation)}
\operation{+suprimerLien(Module, Etudiant)}
\operation{+suprimerLien(Lien)}

\operation{+getLien(Module, Etudiant):Lien}
\operation{+getLiens(Etudiant):Set<Lien>}
\operation{+getLiens(Module):Set<Lien>}

\operation{+getModules(Etudiant):Set<Module>}
\operation{+getEtudiants(Module):Set<Etudiant>}

\operation{+\underline{getInstance()} : AssociationNotation}
\end{singleton}
\begin{class}[text width=4.4cm]{Lien}{7,0}
\attribute{-module : Module}
\attribute{-etudiant : Etudiant}
\attribute{-note : Notation}
\operation{+Lien(Module, Etudiant)}
\operation{+getModule():Module}
\operation{+getEtudiant():Etudiant}
\operation{+getNotation():Notation}
\operation{+setNotation(note : Notation)}
\end{class}
\end{tikzpicture}
\caption{Diagramme de classe de \texttt{AssociationNotation}\label{classe_association_notation}}
\end{figure}

\Question Implémenter l'association «~\texttt{Notation}~» entre \texttt{Etudiant} et \texttt{Module} en utilisant  
l'objet d'association \texttt{AssociationNotation}.
Copier la classe \texttt{testJDBC} dans la nouvelle classe \texttt{testAsso2}.
Modifier le code de cette classe pour charger toutes les 
notes des différents étudiants aux différents modules dans l'objet d'association \texttt{AssociationNotation}. 
Pour simplifier les traitements, penser à charger l'ensemble des étudiants et des modules à l'avance.
Afficher les étudiants et leurs notes pour le module 'ACSI'.

\paragraph{Promotion d'une association en classe :}
La dernière approche présentée a pour objectif de simplifier le diagramme de classe pour contourner le problème 
des associations trop complexes à matérialiser. Comme nous venons de le voir, implémenter une association bidirectionnelle 
non-hiérarchique demande un travail important. Généralement lorsque l'on rencontre des associations $n$-aires (avec $n>2$),
l'une des techniques employées est de promouvoir cette association en une classe. Celle-ci sera liée par une association 
hiérarchique à chacune des classes participant à l'ancienne association. Cette modification du diagramme de classe modifie 
aussi partiellement sa sémantique. En effet, la contrainte d'unicité de l'association n'est plus vérifiée structurellement, 
la responsabilité de cette contrainte revient au code de l'utilisateur. Il faudra en être conscient 
avant de faire le choix d'utiliser cette solution.

Dans notre base de données «~Gestion Pédagogique~» il n'y a qu'une seule association ternaire : \texttt{Enseignement}(\emph{cf.} 
figure~\ref{uml_ternaire1}). Elle sera donc notre support pour mettre en pratique cette technique. 
La figure~\ref{uml_ternaire2} montre les modifications à apporter à cette association pour la promouvoir en classe. 

\begin{figure}\centering

\begin{tikzpicture}[]

\begin{class}[text width=2cm]{Module}{0,0}
\end{class}

\begin{class}[text width=2cm]{Prof}{10,0}
\end{class}

\begin{class}[text width=2cm]{Etudiant}{5,-4.5}
\end{class}

\ternaryassociation%
{Module.south}%
{Prof.south}%
{Etudiant}%
{Enseignement}
\end{tikzpicture}
\caption{Diagramme de classe de l'association ternaire \texttt{Enseignement}\label{uml_ternaire1}}
\end{figure}

\begin{figure}\centering

\begin{tikzpicture}[]

\begin{class}[text width=2cm]{Module}{0,0}
\end{class}

\begin{class}[text width=2cm]{Prof}{10,0}
\end{class}

\begin{class}[text width=2cm]{Etudiant}{5,-5}
\end{class}

\begin{class}[text width=2.7cm]{Enseignement}{5,-2}
\end{class}


\draw [color=\umldrawcolor,text=\umltextcolor] 
(Module.south east) to 
node[very near end, below]{*}
node [above]{$A_1$}
node[very near start, below]{1}  
(Enseignement);

\draw [color=\umldrawcolor,text=\umltextcolor] 
(Prof.south west) to 
node[very near end, below]{*}
node [above]{$A_2$}
node[very near start, below]{1}  
(Enseignement);

\draw [color=\umldrawcolor,text=\umltextcolor] 
(Etudiant) to 
node[near end, right]{*}
node [left]{$A_3$}
node[near start, right]{1}  
(Enseignement);

\end{tikzpicture}
\caption{Diagramme de classe de l'association ternaire \texttt{Enseignement}\label{uml_ternaire2}}
\end{figure}    

\Question Implémenter l'association «~\texttt{Enseignement}~» entre \texttt{Etudiant}, \texttt{Module} et \texttt{Prof} 
en utilisant le diagramme de classe donné par la figure~\ref{uml_ternaire2}. 
Modifier chacune des classes participantes pour que les associations $A_i$ soient navigables dans les deux sens. 
Copier la classe \texttt{testJDBC} dans la nouvelle classe \texttt{testAsso3}.
Modifier le code de cette classe pour charger tous les enseignements. 
Afficher tous les enseignements suivis par les étudiants du groupe 1.

\subsection{Construction de la couche d'accès aux données}
Les paragraphes précédents ont présenté comment construire le modèle objet miroir du modèle relationnel.
L'objectif est maintenant d'écrire le code permettant de faire communiquer ces deux modèles.
Les questions ont mis en évidence la difficulté (et l'aspect répétitif) d'écrire un tel code avec JDBC . 
Utiliser directement JDBC à chaque accès aux données produirait deux effets très négatifs : 
\begin{itemize}
 	\item Une pollution importante du code métier par du code JDBC. Cela implique donc une moins 
 	grande lisibilité du code et ainsi un risque d'erreur plus important.
 	\item Une moins grande indépendance vis à vis du SGBD. L'intrication forte entre code métier et 
 	code d'accès au données rend le changement de SGBD (par exemple le remplacement de Oracle par Postgres) 
 	très délicat voir impossible.
\end{itemize} 
Pour éviter ces problèmes, nous allons construire une couche dédiée à l'accès aux données 
qui utilisera le pattern DAO\footnote{\url{http://java.sun.com/blueprints/corej2eepatterns/Patterns/DataAccessObject.html}} (Data Access Object).
Cette couche encapsulera tous les accès à la source de données. Les autres parties de l'application 
utiliseront uniquement les objets de cette couche pour gérer la persistance. Elle 
sera donc une sorte d'abstraction du modèle de données indépendante de la solution de stockage 
des données. La couche DAO contiendra au moins autant de classes de DAO que d'entités du MCD (classe d'objet métier) 
\footnote{L'écriture et la maintenance d'une telle couche est donc une opération généralement fastidieuse. 
C'est l'une des raisons pour lesquelles les solutions de persistance actuelles génèrent 
automatiquement une grande partie du code (Java et/ou \textsc{Sql}).}.

\begin{figure}\centering\footnotesize
\begin{tikzpicture}
\begin{singleton}[text width=7cm]{DAOEtudiant}{0,0}
\attribute{-\underline{instance} : DAOEtudiant}

\operation{+insert(Etudiant):Etudiant}
\operation{+delete(Etudiant):boolean}
\operation{+update(Etudiant):boolean}

\operation{+getById(numEt:int):Etudiant}

\operation{+findAll():List<Etudiant>}
\operation{+findByNom(NomEt:String):List<Etudiant>}
\operation{+findByGroupe(Groupe:int):List<Etudiant>}
\operation{+findByAnnee(Annee:int):List<Etudiant>}

\operation{+computeMoyennePonderee(Etudiant):float}
\operation{+computeNbEtudiant():int}

\operation{+\underline{getInstance()} : DAOEtudiant}
\end{singleton}
\end{tikzpicture}
\caption{Diagramme de classe de \texttt{DAOEtudiant}\label{dao_etudiant}}
\end{figure}

La source de données étant une ressource unique, il n'existera qu'une seule instance de chacune 
des classes de DAO. Elles devront donc toutes être des singletons (\emph{cf.} classes 
\texttt{ConnexionUnique} et \texttt{AssociationNotation} ). Chacune d'elles devra contenir 
des méthodes pour effectuer les 4 opérations de base pour la persistance des données : 
\textit{créer, récupérer, mettre à jour et supprimer}\footnote{Généralement désigné par l'acronyme anglais CRUD pour 
\textit{Create, Retrieve, Update et Delete}}. Par convention, chacune des classes de DAO devra être 
nommée par "\texttt{DAO}" suivi du nom de la classe métier associée. La figure~\ref{dao_etudiant} 
décrit la classe \texttt{DAOEtudiant} qui est le DAO associé à la classe d'objet métier \texttt{Etudiant}.
Cette classe est constituée des méthodes suivantes :
\begin{itemize}
	\item \texttt{insert} qui à pour objectif de créer un nouvel étudiant dans la base de données.
	L'identifiant d'un tuple ne pouvant être connu avant son insertion, cette méthode retourne 
	une copie de l'objet métier passé en paramètre avec un identifiant définitif. L'identité d'un 
	objet dépendant uniquement de l'identifiant, un objet métier créé localement avec le 
	constructeur par défaut (objet temporaire sans identité propre du point de vue de 
	\texttt{equals()} et \texttt{hashCode()}) ne devra participer à aucune association 
	avant d'être inséré dans la base avec cette méthode\footnote{Ces objets sans identité jouent le rôle des objets 
	de transfert de données (\textit{Data Transfer Object}) du pattern DAO original.}.

	\item \texttt{update} qui prend un objet métier en paramètre et essaie faire la mise à jour 
	dans la base de données. La valeur retournée par cette méthode indique si la mise à jour 
	a pu avoir lieu.
	
  \item \texttt{delete} qui prend un étudiant en paramètre et essaie de le supprimer de la base 
  de données. La valeur retournée par cette méthode indique si la suppression a pu avoir lieu.
  
  \item les \texttt{get} qui constituent, avec les \texttt{find}, les méthodes 
  de récupération des données. Les paramètres passés à ces méthodes permettent de récupérer
  uniquement les tuples satisfaisants certains critères. La différence entre ces deux 
  familles de méthodes est que les \texttt{get} doivent retourner exactement un seul 
  résultat alors que les \texttt{find} peuvent en retourner plusieurs.
  
  \item les \texttt{compute} qui, comme leur nom l'indique, ont pour objectif d'effectuer des 
  calculs sur les étudiants. La plupart du temps (sauf si le calcul demande de ne rapatrier 
  aucune donnée) on préférera, pour des raisons d'efficacité, le faire directement dans le \texttt{Sgbd}. 
  Ces méthodes sont donc soit des requêtes \textsc{Sql} agrégatives soit des appels de procédures stockées.	
\end{itemize}

En utilisant \texttt{DAOEtudiant}, la récupération par l'application de l'étudiant d'identifiant 1 dans la base de données 
se déroule comme suit :
\begin{enumerate}
 	\item L'application demande un objet \texttt{Etudiant} correspondant au tuple d'identifiant 1 dans la base de données 
 	à l'unique instance de \texttt{DAOEtudiant}.
  \item L'objet \texttt{DAOEtudiant} récupère cette demande (méthode \texttt{getByID(1)} ) et il s'occupe d'exécuter la 
  requête SQL avec JDBC.
  \item Le \textsc{Sgbd} interprète la requête SQL et retourne le résultat attendu (s'il existe).
  \item L'objet \texttt{DAOEtudiant} récupère ces informations.
  \item L'objet \texttt{DAOEtudiant} instancie un objet \texttt{Etudiant} avec les données récupérées.
  \item Enfin, l'objet \texttt{DAOEtudiant} retourne l'instance de l'objet \texttt{Etudiant}.
\end{enumerate}

Cette séquence d'opération illustre bien le rôle central de l'objet DAO dans l'accès aux données. Les opérations de mise 
à jour et de suppression se dérouleront à peu près de la même manière. Pour l'insertion d'un nouveau tuple, il faudra 
d'abord créer un objet sans identité (avec le constructeur par défaut) puis appeler la méthode \texttt{insert()} qui nous 
retournera notre objet définitif (avec un identifiant valide). Le code ci-dessous illustre l'utilisation typique du DAO 
pour l'ajout d'un nouvel étudiant et sa modification :
\begin{lstlisting}[language=java]
public class Main {
  public static void main(String[] args){
    DAOEtudiant dao = DAOEtudiant.getInstance();
    Etudiant e = new Etudiant();//e est un Etudiant temporaire
    e.setNom("Dupont");
    e.setPrenom("Paul");
    e.setCp("13100");
    e.setVille("Aix-en-Provence");
    e.setAnnee(1);//Modification des attributs de e 
    e.setGroupe(5);
    e = dao.insert(e);//e referencie maintenant un Etudiant definitif
    ...
    e.setAnnee(2);// Modification des attributs de e 
    e.setGroupe(3);
    ...
    boolean updateOk = dao.update(e);//Sauvegarde des modifications
    ...
  }
}
\end{lstlisting}
Tous les DAO de notre application ont un certain nombre de méthodes communes. Pour améliorer l'indépendance du code client
vis à vis de la couche de persistance, nous ajoutons une interface \texttt{DAO} que tous les objets DAO devront implémenter. 
Les objets métiers dépendront ainsi d'une interface et non d'une implémentation particulière. La figure~\ref{dao_gestion_peda}
donne le diagramme de classe de l'ensemble des DAO de l'application gestion pédagogique. Dans sa version complète, le 
pattern présenté utilise des \texttt{Factory} pour améliorer encore la modularité de la couche de persistance.  


\begin{figure}\centering\footnotesize
\begin{tikzpicture}
\begin{interface}[text width=4cm]{DAO<T>}{6,0}
\operation{+insert(T):T}
\operation{+delete(T):boolean}
\operation{+update(T):boolean}
\operation{+getById(numT:int):T}
\operation{+findAll():List<T>}
\end{interface}

\begin{interface}[text width=2.6cm]{DAONotation}{0,-5}
\end{interface}

\begin{class}[text width=2.6cm]{DAONotationI}{0,-8}
\end{class}

\begin{interface}[text width=2.5cm]{DAOEtudiant}{3,-5}
\end{interface}

\begin{class}[text width=2.5cm]{DAOEtudiantI}{3,-8}
\end{class}


\begin{interface}[text width=2.3cm]{DAOModule}{6,-5}
\end{interface}

\begin{class}[text width=2.3cm]{DAOModuleI}{6,-8}
\end{class}


\begin{interface}[text width=2.3cm]{DAOProf}{8.7,-5}
\end{interface}

\begin{class}[text width=2.3cm]{DAOProfI}{8.7,-8}
\end{class}


\begin{interface}[text width=3.2cm]{DAOEnseignement}{12,-5}
\end{interface}

\begin{class}[text width=3.3cm]{DAOEnseignementI}{12,-8}
\end{class}


\draw [umlcd style inherit line] (DAO<T>) -- (DAONotation);
\draw [umlcd style inherit line] (DAO<T>) -- (DAOEtudiant);
\draw [umlcd style inherit line] (DAO<T>) -- (DAOModule);
\draw [umlcd style inherit line] (DAO<T>) -- (DAOProf);
\draw [umlcd style inherit line] (DAO<T>) -- (DAOEnseignement);

\draw [umlcd style implement line] (DAONotation) -- (DAONotationI);
\draw [umlcd style implement line] (DAOEtudiant) -- (DAOEtudiantI);
\draw [umlcd style implement line] (DAOModule) -- (DAOModuleI);
\draw [umlcd style implement line] (DAOProf) -- (DAOProfI);
\draw [umlcd style implement line] (DAOEnseignement) -- (DAOEnseignementI);

\end{tikzpicture}
\caption{Diagramme de classe de la couche DAO de l'application gestion pédagogique\label{dao_gestion_peda}}
\end{figure}

\Question
Implémenter les classes \texttt{DAOEtudiant}\footnote{Pour ceux qui n'auraient pas terminé la première partie du TP, un 
squelette des classes métiers est disponible à l'adresse suivante : \url{http://allegro.iut.univ-aix.fr/~nedjar/testJdbc.tar.bz2}} 
dont le diagramme de classe incomplet vous est donné par la figure~\ref{dao_etudiant}. 
\Question
Implémenter toutes les classes \texttt{DAO} en prenant en compte intéligement les associations existant entre les 
différentes classes métiers. Copier la classe \texttt{testJDBC} dans la nouvelle classe \texttt{testDAO}. Modifier le 
code de celle-ci pour que sa boucle principale récupère tous les étudiants de deuxième années, les affiche, puis 
augmente toutes leurs notes pour le module « ACSI » d'un point et enfin sauvegarde les résultats dans la base.

\bibliographystyle{apalike}
\bibliography{biblio}
\end{document}

